\chapter{Introduction}

Moving meshes are required in various problems in computational fluid dynamics (CFD), such as aeroelasticity, aerodynamic shape optimization \cite{appl:opt} and some kinds of multi-phase/multi-material flow simulations. Mesh movement may also be used in generating curved meshes for spatially high-order computational methods \cite{curve:persson}. In general, good properties of a mesh movement scheme are as follows.
\begin{enumerate}
\item It should be robust. In many kinds of meshes, such as meshes required for viscous flow computations, even small boundary movements can invalidate the mesh. The scheme should be able to preserve mesh validity, at the very least.
\item Mesh quality should be preserved to a great extent. Elements should not become highly skewed or mis-shaped after movement, as this can impact flow computations on the deformed mesh. We discuss skew and shape of elements later in the report.
\item It should be computationally inexpensive. Many applications need very fast mesh movement schemes as they require the mesh to be moved many times during the simulation, possibly every time step.
\end{enumerate}
Some mesh movement schemes that we present here satisfy only one or two of the above criteria. Our aim is to find a scheme that satisfies all three criteria well. We give a review of various mesh-movement methods in chapter 2.

With the rising popularity of high-order computational methods in the aerospace community, robust techniques to obtain a high-order representation of the geometry, ie., the boundary of the domain, have become important. High-order approximation of the boundary is required to attain high-order accuracy \cite{curve:geomacc}, and in some cases it may be crucial to even preserve qualitative correctness of the flow \cite{appl:dgeuler}. One technique of high-order boundary approximation is to produce high-order meshes, otherwise called curvilinear or curved meshes. Another technique in this regard is isogeometric analysis \cite{isogeometric} where CAD data is directly used in analysis. In this work, we deal with curved unstructured mesh generation, using either CAD data if available, or only the linear mesh data. If only linear mesh data is available, some kind of high-order boundary reconstruction is used to compute a higher order approximation, and the mesh is then curved according to this reconstructed boundary. In chapter 3, we describe the method of curved mesh generation adopted for this work.

Finally, results for both mesh movement and curved mesh generation are presented in chapter 4. We conclude in chapter 5 with a succint comparison of the methods and give some more details of our implementation in the appendices.
