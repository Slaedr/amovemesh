\chapter{Conclusions and Future Work}

A lot of choice is available for mesh-movement techniques, but only a few will be good for a given application. We conclude that interpolation methods are generally well-suited for unsteady simulations that require mesh movement, as they are very fast. When deformations are relatively small and not highly rotational, DGM is a good choice while when larger and more general deformations are needed, DGRBF is best. The pure RBF method, while usually giving good robust results, is more expensive than other interpolation methods considered here. 

RBF, and to some extent linear elasticity methods, are found to be effective for curved mesh generation, where computational cost is less of an issue. We have used RBF for curved mesh generation with good results for certain viscous flow cases.

One direction for future work is to complete a surface reconstruction procedure in 3D. For some meshes, the requirement of global $C^2$ continuity is too restrictive. Also, a procedure similar to that described in section \ref{subsec:spline2d} would be quite expensive for surfaces in $\mathbb{R}^3$. We therefore consider local fittings of 2D Taylor polynomials at every boundary vertex, described in \cite{sr:jiaowang} as ``Weighted Averaging of Local Fittings" (WALF). In their paper, Jiao and Wang fit local 2D Taylor polynomials to each vertex of the surface mesh. The coefficients, that is, the derivatives of a local height function in a local coordinate system centered at the vertex, are solved for using vertex position data from a neighborhood of that vertex. It is ensured that there are more neighboring points being considered for data than the number of unknowns to solve for, thereby obtaining an over-determined system of equations. This is solved by a weighted least-squares approach. This is claimed to work well for both smooth surfaces and surfaces with $C^1$ discontinuities (ridges, corners etc.). For the latter case, additional preprocessing of the linear surface mesh is required to detect discontinuities \cite{sr:discontinuities}.
