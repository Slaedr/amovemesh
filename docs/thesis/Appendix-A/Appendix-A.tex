\chapter{Delaunay tessellation algorithm}

Here, the exact algorithm used for the Delaunay tessellation in the context of Delaunay graph mapping (DGM) methods is explained. The method broadly follows the one used by Bowyer \cite{bowyer}. 

The Delaunay process maximizes the minimum angle in a simplex. Implementation of the ``Delaunay kernel" is done such that orientation of elements and faces is preserved.

\chapter[Spline reconstruction]{Spline reconstruction of a piecewise-linear boundary}
\label{app:spline}

The exact process used to reconstruct a twice-differentiable ($C^2$) boundary from the piecewise-linear boundary of a linear 2D mesh is described. Each one-dimensional boundary facet of the two-dimensional mesh is reconstructed into a cubic spline. The basic framework of the method is taken from \cite{sr:wolframspline}. We assume there are $N$ boundary facets and thus $N+1$ boundary nodes.

We want to reconstruct the $i$th boundary facet into a curve in $\mathbb{R}^2$ as
\begin{eqnarray}
\bld{r}_i(t) &=& \bld{a}_i + \bld{b}_i t + \bld{c}_it^2 + \bld{d}_it^3, \quad t \in [0,1]
%\bld{r}_i'(t) &=& \bld{b}_i + 2\bld{c}_it + 3\bld{d}_it^2 \\
%\bld{r}_i''(t) &=& 2\bld{c}_i + 6\bld{d}_it
\end{eqnarray}
$t=0$ corresponds to the starting point of the boundary facet while $t=1$ corresponds to the ending point. We impose $C^0$, $C^1$ and $C^2$ continuity of the curve to form a square linear system. That is, for all $i \in \{0,1,...,N\}$
\begin{eqnarray}
\bld{r}_i(0) &=& \bld{R}_i \\
\bld{r}_i(1) &=& \bld{R}_{i+1} \\
\bld{r}_i'(0) &=& \bld{r}_{i-1}'(1) \\
\bld{r}_i'(1) &=& \bld{r}_{i+1}'(0) \\
\bld{r}_i''(0) &=& \bld{r}_{i-1}''(1) \\
\bld{r}_i''(1) &=& \bld{r}_{i+1}''(0)
\end{eqnarray}
where $\bld{R}_i$ denotes the coordinates of the $i$th boundary point. The ``boundary conditions'' are, in case of a closed curve,
\begin{eqnarray}
\bld{R}_0 &=& \bld{R}_{N+1} \\
\bld{r}_0'(0) &=& \bld{r}_N'(1) \\
\bld{r}_0''(0) &=& \bld{r}_N''(1)
\end{eqnarray}
and in case of an open curve
\begin{eqnarray}
\bld{r}_0''(0) &=& 0 \\
\bld{r}_N''(1) &=& 0.
\end{eqnarray}

For assembling the 3 systems of equations for the x-, y- and z-components of the spline curve, we define $D_i := r_i'(0) \, \forall \, i \in \{1,2,...,N\}$ and $D_{N+1} := r_N'(1)$ where $r$ is $x$, $y$ and $z$ in turn. We express the coefficients $a_i$, $b_i$, $c_i$ and $d_i$ in terms of the $D_i$ and $R_i$ (the coordinates of boundary nodes). Finally, we obtain, for an open curve
\begin{equation}
\begin{bmatrix}
2 & 1 & & & & & \\
 & 1 & 4 & 1 & & & \\
 & & \ddots & \ddots & \ddots & &\\
 & & & & 1 & 4 & 1 \\
 & & & & & 1 & 2
\end{bmatrix}
\begin{bmatrix}
D_1 \\
D_2 \\
\vdots \\
D_N \\
D_{N+1}
\end{bmatrix}
=
\begin{bmatrix}
3(r_2 - r_1) \\
3(r_3 - r_1) \\
\vdots \\
3(r_{N+1}-r_{N-1}) \\
3(r_{N+1} - r_N)
\end{bmatrix}.
\end{equation}
After a similar derivation, for a closed curve we obtain
\begin{equation}
\begin{bmatrix}
4 & 1 & & & & 1 & 0 \\
& 1 & 4 & 1 & & & \\
& & \ddots & \ddots & \ddots & &\\
& & & & 1 & 4 & 1 \\
0 &1 & & & & 1 & 4
\end{bmatrix}
\begin{bmatrix}
D_1 \\
D_2 \\
\vdots \\
D_N \\
D_{N+1}
\end{bmatrix}
=
\begin{bmatrix}
3(r_2 - r_{N}) \\
3(r_3 - r_1) \\
\vdots \\
3(r_{N+1}-r_{N-1}) \\
3(r_2 - r_N)
\end{bmatrix}.
\end{equation}
